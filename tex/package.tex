\usepackage{lmodern}
\usepackage[utf8]{inputenc}
\usepackage[T1]{fontenc}
\usepackage[french]{babel}

\usepackage{amsmath}
\usepackage{amsfonts}
\usepackage{amssymb}
\usepackage{mathrsfs}

\usepackage{multicol}

\newtheorem{theoreme}{Théorème}[section]
\newtheorem{corollaire}[theoreme]{Corollaire}
\newtheorem{lemme}{Lemme}[section]
\newtheorem{definition}{Définition}[section]
\newtheorem{propriete}{Propriété}[section]
\newtheorem{exemple}{Exemple}[section]

\newenvironment{demo}{
\emph{Démonstration}
\small 
\setlength{\parindent}{0pt} 
\begin{list}{}{%
\setlength{\leftmargin}{10pt}
\setlength{\labelwidth}{0pt}
\setlength{\labelsep}{0pt}
}
\item
} 
{
\hfill $ \Box $
\end{list}
\normalsize
}

\newenvironment{remarque}{
\textsc{Remarque}
} 
{
}

\usepackage{hyperref}

%biblio environment
\newcommand{\book}[4]{\item \textsc{#1}. \textit{#2}. #3, #4}

\newcommand{\article}[5]{
\item \textsc{#1}. #2: \textit{#3} (#4), DOI: \url{#5}
}

\newcommand{\lecture}[5]{
\item \textsc{#1}. \textit{#2}, #3 (#4), \url{#5}
}

\newenvironment{biblio}{
\chapter*{Bibliographie}
\begin{enumerate}[label=(\arabic*)]
\begin{large}
}
{
\end{large}
\end{enumerate}
}

%\usepackage{xcolor}
\usepackage{tikz}

%\usepackage{caption}
