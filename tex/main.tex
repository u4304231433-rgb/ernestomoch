\documentclass[a4paper, 12pt] {report}
\usepackage{fontspec}

\usepackage{blindtext}
\usepackage{csvsimple}
\usepackage{ifthen}


\usepackage{fancyhdr}
\pagestyle{fancy}

\usepackage{zref-savepos}
\usepackage{zref-user}

\usepackage{fontspec}

\usepackage{xstring}
\usepackage{textcase}

\usepackage{hyperref}

\usepackage{environ}


\newcommand{\NormalizeChar}[1]{%
  \IfSubStr{éèêÉÈÊ}{#1}{E}{%
    \IfSubStr{àâÀÂ}{#1}{A}{%
      \IfSubStr{ùûÙÛ}{#1}{U}{%
        \IfSubStr{îÎ}{#1}{I}{%
          \IfSubStr{ôÔ}{#1}{O}{%
            	\MakeUppercase{#1}%
          }%
        }%
      }%
    }%
  }%
}

\newcommand{\NormalizeCharErnestien}[1]{%
  \IfSubStr{éèêÉÈÊ}{#1}{E}{%
    \IfSubStr{àâÀÂ}{#1}{A}{%
      \IfSubStr{ùûÙÛ}{#1}{U}{%
        \IfSubStr{îÎ}{#1}{I}{%
          \IfSubStr{ôÔ}{#1}{O}{%
          	\IfSubStr{cC}{#1}{H}{%
            		\MakeUppercase{#1}%
          	}%
          }%
        }%
      }%
    }%
  }%
}

\newcommand{\Normalize}[1]{%
  % Conversion des minuscules accentuées
  \StrSubstitute{#1}{é}{E}[\tempA]%
  \StrSubstitute{\tempA}{è}{E}[\tempB]%
  \StrSubstitute{\tempB}{ê}{E}[\tempC]%
  \StrSubstitute{\tempC}{ë}{E}[\tempD1]%
  \StrSubstitute{\tempD1}{à}{A}[\tempD2]%
  \StrSubstitute{\tempD2}{â}{A}[\tempD3]%
  \StrSubstitute{\tempD3}{ä}{A}[\tempD4]%
  \StrSubstitute{\tempD4}{ù}{U}[\tempD5]%
  \StrSubstitute{\tempD5}{û}{U}[\tempD6]%
  \StrSubstitute{\tempD6}{ü}{U}[\tempD7]%
  \StrSubstitute{\tempD7}{î}{I}[\tempD8]%
  \StrSubstitute{\tempD8}{ï}{I}[\tempD9]%
  \StrSubstitute{\tempD9}{ô}{O}[\tempD10]%
  \StrSubstitute{\tempD10}{ö}{O}[\tempD11]%
  \StrSubstitute{\tempD11}{ç}{C}[\tempD12]%
  % Conversion des majuscules accentuées
  \StrSubstitute{\tempD12}{É}{E}[\tempE1]%
  \StrSubstitute{\tempE1}{È}{E}[\tempE2]%
  \StrSubstitute{\tempE2}{Ê}{E}[\tempE3]%
  \StrSubstitute{\tempE3}{Ë}{E}[\tempE4]%
  \StrSubstitute{\tempE4}{À}{A}[\tempE5]%
  \StrSubstitute{\tempE5}{Â}{A}[\tempE6]%
  \StrSubstitute{\tempE6}{Ä}{A}[\tempE7]%
  \StrSubstitute{\tempE7}{Ù}{U}[\tempE8]%
  \StrSubstitute{\tempE8}{Û}{U}[\tempE9]%
  \StrSubstitute{\tempE9}{Ü}{U}[\tempE10]%
  \StrSubstitute{\tempE10}{Î}{I}[\tempE11]%
  \StrSubstitute{\tempE11}{Ï}{I}[\tempE12]%
  \StrSubstitute{\tempE12}{Ô}{O}[\tempE13]%
  \StrSubstitute{\tempE13}{Ö}{O}[\tempE14]%
  \StrSubstitute{\tempE14}{Ç}{C}[\tempE]%
  % Conversion manuelle des minuscules restantes
  \StrSubstitute{\tempE}{a}{A}[\tempF]%
  \StrSubstitute{\tempF}{b}{B}[\tempG]%
  \StrSubstitute{\tempG}{c}{C}[\tempH]%
  \StrSubstitute{\tempH}{d}{D}[\tempI]%
  \StrSubstitute{\tempI}{e}{E}[\tempJ]%
  \StrSubstitute{\tempJ}{f}{F}[\tempK]%
  \StrSubstitute{\tempK}{g}{G}[\tempL]%
  \StrSubstitute{\tempL}{h}{H}[\tempM]%
  \StrSubstitute{\tempM}{i}{I}[\tempN]%
  \StrSubstitute{\tempN}{j}{J}[\tempO]%
  \StrSubstitute{\tempO}{k}{K}[\tempP]%
  \StrSubstitute{\tempP}{l}{L}[\tempQ]%
  \StrSubstitute{\tempQ}{m}{M}[\tempR]%
  \StrSubstitute{\tempR}{n}{N}[\tempS]%
  \StrSubstitute{\tempS}{o}{O}[\tempT]%
  \StrSubstitute{\tempT2}{p}{P}[\tempU]%
  \StrSubstitute{\tempU}{q}{Q}[\tempV]%
  \StrSubstitute{\tempV}{r}{R}[\tempW]%
  \StrSubstitute{\tempW}{s}{S}[\tempX]%
  \StrSubstitute{\tempX}{t}{T}[\tempY]%
  \StrSubstitute{\tempY}{u}{U}[\tempZ]%
  \StrSubstitute{\tempZ}{v}{V}[\tempAA]%
  \StrSubstitute{\tempAA}{w}{W}[\tempAB]%
  \StrSubstitute{\tempAB}{x}{X}[\tempAC]%
  \StrSubstitute{\tempAC}{y}{Y}[\tempAD]%
  \StrSubstitute{\tempAD}{z}{Z}[\NormalizedResult]%
}

\newcommand{\IfLike}[2]{%
  \Normalize{#1}%
  \edef\NormText{\NormalizedResult}%
  \Normalize{#2}%
  \edef\NormMot{\NormalizedResult}%
  \IfSubStr{\NormText}{\NormMot}
}


\newcommand{\ifequal}[4]{%
  \ifthenelse{\equal{#1}{#2}}{%
    #3%
  }{%
    #4%
  }%
}

\newcommand{\ern}[1]{%
\StrSubstitute{#1}{--}{\textendash}[\temp]%
{\small\fontspec{ErnestFont}{\temp}}%
}

\newcommand{\erntrad}[1]{%
\StrSubstitute{#1}{--}{\textendash}[\temp]%
{\small\fontspec{ErnestFont}{\temp}} (\temp)%
}

\newcommand{\largeern}[1]{%
\StrSubstitute{#1}{--}{\textendash}[\temp]%
{\normalsize\fontspec{ErnestFont}{\temp}}%
}

\newcommand{\largeerntrad}[1]{%
\StrSubstitute{#1}{--}{\textendash}[\temp]%
{\normalsize\fontspec{ErnestFont}{\temp}} (\temp)%
}

\newcommand{\hugeern}[1]{%
\StrSubstitute{#1}{--}{\textendash}[\temp]%
{\huge\fontspec{ErnestFont}{\temp}}%
}

\newcommand{\comma}{,}
\renewcommand{\backslash}{/}


\NewEnviron{ernestienenv}{%
\StrSubstitute{\BODY}{--}{\textendash}[\temp]%
\small\fontspec{ErnestFont}\temp\;}

\NewEnviron{ernestientrad}{ %
\StrSubstitute{\BODY}{--}{\textendash}[\temp]%
{\small\fontspec{ErnestFont}\temp} (\temp)\;%
}


\newcommand{\setvariable}[2]{%
  \expandafter\edef\csname var@{#1}\endcsname{#2}%
}

\newcommand{\getvariable}[1]{%
  \csname var@{#1}\endcsname%
}


\newcommand{\thelastpage}{}

\renewcommand{\headrulewidth}{0pt}
%\fancyfoot[C]{}
%\fancyhead[C]{\thepage}
\fancyhead[R]{\getvariable{lastword\thepage}}%
\fancyhead[L]{\getvariable{firstword\thepage}}%

\makeatletter
\newcommand{\getpage}[1]{%
  \zref@extractdefault{elem:#1}{page}{0}%
}
\makeatother

\setvariable{lastletter}{0}
\setvariable{actualletter}{}


\newcommand\setword[1]{%
  \StrLeft{#1}{1}[\premierCaractere]%
  \IfLike{\premierCaractere}{\getvariable{lastletter}}{}{%
    \linebreak%
    \textbf{\Huge{\NormalizeChar{\premierCaractere}}}%
    \nopagebreak\linebreak\nopagebreak%
  }%
  \setvariable{lastletter}{\premierCaractere}%
%
  \zlabel{elem:#1}%
  \hypertarget{#1}{\label{#1}}%
%
  \ifthenelse{\equal{\getpage{#1}}{\thelastpage}}{%
    \setvariable{lastword\getpage{#1}}{#1}%
  }{%
    \setvariable{firstword\getpage{#1}}{#1}%
    \setvariable{lastword\getpage{#1}}{#1}%
    \edef\thelastpage{\getpage{#1}}%
  }%
}


\newcommand\setwordernestien[1]{%
  \StrLeft{#1}{1}[\premierCaractere]%
  \StrLeft{#1}{2}[\2premierCaractere]%
  \IfLike{\premierCaractere}{\getvariable{lastletter}}{}{%
    \linebreak%
    \textbf{\Huge{\fontspec{ErnestFont}\NormalizeCharErnestien{\premierCaractere}}}%
    \nopagebreak\linebreak\nopagebreak%
  }%
  \setvariable{lastletter}{\premierCaractere}%
%
  \zlabel{elem:#1}%
%
  \ifthenelse{\equal{\getpage{#1}}{\thelastpage}}{%
    \setvariable{lastword\getpage{#1}}{#1}%
  }{%
    \setvariable{firstword\getpage{#1}}{#1}%
    \setvariable{lastword\getpage{#1}}{#1}%
    \edef\thelastpage{\getpage{#1}}%
  }%
}

\newcommand\setwordsansethym[1]{%
  \StrLeft{#1}{1}[\premierCaractere]%
  \StrLeft{#1}{2}[\2premierCaractere]%
  \IfLike{\premierCaractere}{\getvariable{lastletter}}{}{%
    \linebreak%
    \textbf{\Huge{\NormalizeChar{\premierCaractere}}}%
    \nopagebreak\linebreak\nopagebreak%
  }%
  \setvariable{lastletter}{\premierCaractere}%
%
  \zlabel{elem:{#1}_sansethym}%
%
  \ifthenelse{\equal{\getpage{{#1}_sansethym}}{\thelastpage}}{%
    \setvariable{lastword\getpage{{#1}_sansethym}}{#1}%
  }{%
    \setvariable{firstword\getpage{{#1}_sansethym}}{#1}%
    \setvariable{lastword\getpage{{#1}_sansethym}}{#1}%
    \edef\thelastpage{\getpage{{#1}_sansethym}}%
  }%
}

\usepackage{lmodern}
\usepackage[utf8]{inputenc}
\usepackage[T1]{fontenc}
\usepackage[french]{babel}

\usepackage{amsmath}
\usepackage{amsfonts}
\usepackage{amssymb}
\usepackage{mathrsfs}

\usepackage{multicol}

\newtheorem{theoreme}{Théorème}[section]
\newtheorem{corollaire}[theoreme]{Corollaire}
\newtheorem{lemme}{Lemme}[section]
\newtheorem{definition}{Définition}[section]
\newtheorem{propriete}{Propriété}[section]
\newtheorem{exemple}{Exemple}[section]

\newenvironment{demo}{
\emph{Démonstration}
\small 
\setlength{\parindent}{0pt} 
\begin{list}{}{%
\setlength{\leftmargin}{10pt}
\setlength{\labelwidth}{0pt}
\setlength{\labelsep}{0pt}
}
\item
} 
{
\hfill $ \Box $
\end{list}
\normalsize
}

\newenvironment{remarque}{
\textsc{Remarque}
} 
{
}

\usepackage{hyperref}

%biblio environment
\newcommand{\book}[4]{\item \textsc{#1}. \textit{#2}. #3, #4}

\newcommand{\article}[5]{
\item \textsc{#1}. #2: \textit{#3} (#4), DOI: \url{#5}
}

\newcommand{\lecture}[5]{
\item \textsc{#1}. \textit{#2}, #3 (#4), \url{#5}
}

\newenvironment{biblio}{
\chapter*{Bibliographie}
\begin{enumerate}[label=(\arabic*)]
\begin{large}
}
{
\end{large}
\end{enumerate}
}

%\usepackage{xcolor}
\usepackage{tikz}

%\usepackage{caption}



\begin{document}

\everymath{\displaystyle}

\title{Dictionnaire Ernestien officiel \\ \largeern{rislojisiêl ernestinbûchg librêg}}
\author{Démocratie Ernestienne Normalienne et Supercool \\ \ern{nuranjêrs ernestinbûchg bizargêst feltanîg}}

\maketitle

%\tableofcontents
\pagebreak

\noindent\textbf{\Huge{Guide du dictionnaire}}
\vspace{1em}

\noindent\textbf{\Large{Sommaire}}

\begin{center}
\hyperlink{grammtitle}{Grammaire et formatage lexical \dotfill \pageref{grammtitle}} \\
\hyperlink{fraerntitle}{Français en ernestien \dotfill \pageref{fraerntitle}} \\
\hyperlink{ernfratitle}{Ernestien en français \dotfill \pageref{ernfratitle}} \\
\hyperlink{fraernethymtitle}{Français en ernestien et éthymologie \dotfill \pageref{fraernethymtitle}}

\end{center}

\vspace{2em}
\noindent\textbf{\Large{Abréviations}}

\begin{center}
\begin{tabular}{r l}
\textbf{adj.} & Adjectif \\
\textbf{adv.} & Adverbe \\
\textbf{litt.} & Littéralement \\
\textbf{n.} & Nom \\
\textbf{ref.} & Référence \\
\end{tabular}
\end{center}

\pagebreak

\hypertarget{grammtitle}{\noindent\textbf{\Huge{Grammaire et formatage lexical}}\label{grammtitle}}

\vspace{1em}

\noindent\textbf{\Large{Structure et accords}}

\noindent L'ernestien est une langue à structure simple \{sujet\} -- \{verbe\} -- \{complément\}. 

\noindent Les adjectifs ou adverbes se placent après les noms ou verbes et ne s'accordent pas.

\noindent Les mots peuvent être formés par concaténation de noms et adjectifs ou noms entre eux. Par exemple, dictionnaire \erntrad{rislojisiêl} est formé par concaténation des mots \erntrad{rîs}, mot et \erntrad{lojisiêl}, application.

\noindent Les noms au pluriel prennent la terminaison \erntrad{--i}

\vspace{1em}

\noindent\textbf{\Large{Règle orthographique spéciale}}

\noindent La dernière voyelle de chaque mot prend un accent circonflexe \erntrad{odform}, sauf pour le mot \erntrad{odform}. Pendant les années bissextiles, l'accent doit être déplacé sur l'avant-dernière voyelle si possible, ou disparaître.

\vspace{1em}

\noindent\textbf{\Large{Règles de formatage lexical}}

\noindent Le formatage des mots suit les règles résumées ci-dessous :

\begin{center}
\begin{tabular}{r l}
\textbf{Nature lexicale ou concept} & Préfixe ou suffixe \\
\hline \\
\textbf{Adjectif, adverbe} & \erntrad{-g} \\
\textbf{Contraire} & \erntrad{mal-} \\
\textbf{Disciplines} & \erntrad{-ôn} \\
\textbf{Fait de} & \erntrad{-f} \\
\textbf{Impératif} & \erntrad{-avekû} \\
\textbf{Lieu} & \erntrad{-ô} \\
\textbf{Le meilleur} & \erntrad{-êst} \\
\textbf{Outil} & \erntrad{-ôl} \\
\textbf{Personne} & \erntrad{-z} \\
\textbf{Rendre qqch \{adj\}} & \erntrad{-glê}
\end{tabular}
\end{center}

\noindent \textbf{Exceptions} --- Un adjectif ou adverbe construit sur un ancien adjectif ou adverbe ne prend pas un nouveau \erntrad{--g} (ex : bizargêst).

\noindent \textbf{Exemples} --- Considérons le mot \erntrad{strûnz} -- pierre. L'adjectif "en pierre" se dit donc \erntrad{strûnzg}, la géologie se dit \erntrad{strunzôn}. Les outils relatifs aux pierres (pioche par exemple) se disent \erntrad{strunzôl} et les lieux (grotte, carrière...) \erntrad{strunzô}.

\noindent L'ernestien étant une langue interprétative, la connaissance du contexte est nécessaire pour déterminer le sens précis des mots.

\vspace{1.5em}

\noindent\textbf{\Large{Règles de conjugaison}}

\begin{center}
\begin{tabular}{r l}
\textbf{Temps} & Conjugaison \\
\hline \\
\textbf{Infinitif} & \{verbe\} \\
\textbf{Présent} & \{verbe\}\ern{q} \\
\textbf{Passé} & \ern{q}\{verbe\} \\
\textbf{Futur} & \{verbe\}\ern{qq} \\
\end{tabular}
\end{center}

\pagebreak

\hypertarget{fraerntitle}{\noindent\textbf{\Huge{Français en ernestien}}\label{fraerntitle}}

\setlength{\columnsep}{30pt}

\begin{multicols}{2}
\noindent
%sort by=frersort.xml,
\csvreader[head to column names]{frer.csv}{Francais=\francais, Ernestien=\ernestien, Etymologie=\etymologie}{%
\noindent\setwordsansethym{\francais}\textbf{\hyperlink{\francais}{\francais}} --- \erntrad{\ernestien}\\}
\vfill
\pagebreak
\end{multicols}

\pagebreak

\hypertarget{ernfratitle}{\noindent\textbf{\Huge{\hugeern{ernestinbûch frêd etiminbûch}}}\label{ernfratitle}}

\begin{multicols}{2}
\noindent
%sort by=frersort.xml,
\csvreader[head to column names]{erfr.csv}{Francais=\francais, Ernestien=\ernestien, Etymologie=\etymologie}{%
\noindent\setwordernestien{\ernestien}\textbf{\erntrad{\ernestien}} --- \hyperlink{\francais}{\francais}\\}
\vfill
\pagebreak
\end{multicols}

\pagebreak

\hypertarget{fraernethymtitle}{\noindent\textbf{\Huge{Français en ernestien \& étymologie}}\label{fraernethymtitle}}

\setlength{\columnsep}{30pt}

\begin{multicols}{2}
\noindent
%sort by=frersort.xml,
\csvreader[head to column names]{frer.csv}{Francais=\francais, Ernestien=\ernestien, Etymologie=\etymologie}{%
\noindent\setword{\francais}\textbf{\francais} --- \erntrad{\ernestien} \nolinebreak\ifequal{\etymologie}{}{}{: \textit{\etymologie}} \\}
\vfill
\pagebreak
\end{multicols}

\end{document}